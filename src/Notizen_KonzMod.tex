%! Author = aurelius
%! Date = 31.07.20

% Preamble
\documentclass{article}
% Packages
\usepackage{amsmath}
\usepackage{fancyhdr}
\usepackage[margin=0.5in]{geometry}
\usepackage{indentfirst}
\usepackage[utf8]{inputenc}
\usepackage[T1]{fontenc}
\usepackage{lastpage}
\usepackage{amsfonts}
\usepackage{mathtools}
\usepackage{ulem}
\usepackage{setspace}
\usepackage{cancel}
\usepackage{xcolor}
\usepackage{tikz}
\usetikzlibrary{trees}
\usepackage[ngerman]{datetime}
\usepackage[ngerman]{babel}
\usepackage{hyphenat}
\usepackage{amssymb}
\hyphenation{Mathe-matik wieder-gewinnen}
\usetikzlibrary{er,positioning}


\usetikzlibrary{shapes.geometric}
\usetikzlibrary{arrows}
\usetikzlibrary{fit}

\tikzstyle{every entity} = []
\tikzstyle{every weak entity} = []
\tikzstyle{every attribute} = []
\tikzstyle{every relationship} = []
\tikzstyle{every link} = []
\tikzstyle{every isa} = []

\tikzstyle{link} = [>=triangle 60, draw, thick, every link]

\tikzstyle{total} = [link, double, double distance=3pt]

\tikzstyle{entity} = [rectangle, draw, black, very thick,
minimum width=6em, minimum height=3em,
every entity]

\tikzstyle{weak entity} = [entity, double, double distance=2pt,
every weak entity]

\tikzstyle{attribute} = [ellipse, draw, black, very thick,
minimum width=5em, minimum height=2em,
every attribute]

\tikzstyle{multi attribute} = [attribute, double, double distance=2pt]

\tikzstyle{derived attribute} = [attribute, dashed]

\tikzstyle{relationship} = [top color=white, bottom color=white, diamond, draw, black, very thick,
minimum width=2em, aspect=1,
every relationship]

\tikzstyle{ident relationship} = [relationship, double, double distance=2pt]


\newcommand{\KW}[1]{\indexspace\textbf{\underline{#1:}} \\}
\newcommand{\Def}[2]{\indexspace\textbf{\underline{#1:}} \\ \\\textbf{Def.} $\rightarrow$ {#2}}
\newcommand{\DApp}[2]{\textbf{#1} $\rightarrow$ {#2}}
\newcommand{\TODO}[1]{\color{green} \textbf{/TODO} - {#1} \color{black}}


\addtolength{\voffset}{1cm}
\addtolength{\textheight}{-0.6cm}
\setlength{\footskip}{0cm}
\cfoot{}
\lhead{Konzeptionelle Modellierung}
\rhead{\today, Seite \thepage.}
\pagestyle{fancy}

% Document
\title{Konzeptionelle Modellierung WS 19/20}
\author{Aurelius Adrian}
\date{\today}

\begin{document}
    \maketitle
    \tableofcontents

    \newpage
    \section{Grundlagen}
    \subsection{Modellierung - Grundbegriffe}
    \Def{Modell}{Ein Modell ist ein zweckgerichtetes vereinfachtes Abbild der Wirklichkeit} \\
    \DApp{Zweck}{Spezifizieren, Konstruieren, Visualisieren u. Dokumentieren von Software-Systemen}
    \subsection{Datenbank - Grundbegriffe}
    \Def{Datenbank}{(EN) A Database is a collection of related Data.} \\
    \DApp{Def.}{(DEU) Eine Datenbank ist eine Sammlung zusammenhängender Daten.} \\
    \Def{Konzeptionelles Datenbankschema}{Representation eines Ausschnitts der realen Welt - "Miniwelt"} \\
    \DApp{Zweck}{Ermöglicht eine gemeinsame Sicht (Anwendungsunabhängigkeit)} \\
    \Def{Datenbank Management System (DBMS)}{Sammlung von Programmen zur Verwaltung einer Datenbank} \\
    \DApp{Zweck}{Erzeugung und Warten von sowie Konsistenter Zugriff auf Datenbanken}
    \Def{Datenbank System (DBS)}{Datenbank + Datenbank Management System (DBMS)}
    \Def{Datenbankanwendung}{Datenbank System (DBS) + Anwendungsprogramme}
    \Def{Datenmodell (DM-Modell)}{Strukturierungsvorschrift wie z.B.: Relationenmodell, Netzwerkmodell, Hierachisches Datenmodell}
    \Def{Datenbankschema}{Beschreibung einer konkreten Datenbank}
    \KW{Vorteile von Datenbanken}
    \begin{itemize}
        \item Vermeidung redundanter Daten
        \item Zentrale Kontrolle der Datenintegrität
        \subitem - Logische Datenintegrität
        \subitem - Physische Datenintegrität
        \item Synchronisation im Mehrbenutzerbetrieb
        \item Fehlertoleranz
    \end{itemize}

    \newpage %tmp

    \KW{Inhalt einer Datenbank} \\
    \DApp{Nutzdaten}{Eigentliche Datenbank} \\
    \DApp{Metadaten}{Daten über die Strukturierung der Nutzdaten}
    
    \KW{Aufbau von Datenbanken} \\
    \tikzstyle{every node}=[draw=black,thick,anchor=west]
    \tikzstyle{selected}=[draw=red,fill=red!30]
    \tikzstyle{optional}=[dashed,fill=gray!50]
    \begin{tikzpicture}[
    grow via three points={one child at (0.5,-0.7) and
    two children at (0.5,-0.7) and (0.5,-1.4)},
    edge from parent path={(\tikzparentnode.south) |- (\tikzchildnode.west)}]
        \node [selected] {Benutzer}
        child {node {Datenbankanwendung}
        child {node {Anwendungsprogramme}}
        child {node {Datenbank System (DBS)}
        child {node {Datenbank Management System (DBMS)}}
        child {node {Datenbank (DB)}}}}
    \end{tikzpicture}
    
    \subsection{3-Schema-Architektur (ANSI/SPARC, 1975)}
    Trennt 3 strikt Arten von Metadaten:
    \Def{1 - Konzeptionelles Schema}{Beschreibt sämtliche Daten auf Logischer Ebene z.B. Patient(Nr., Krankenkasse, Laborwerte)}
    \Def{2 - Externes Schema}{Beschreibt für eine Anwendung relevanten Teil (Ausschnitt) einer Datenbank auf logischer Ebene z.B. für den Arzt - Patient (Nr., Laborwerte) oder für die Krankenhausverwaltung - Patient(Nr., Krankenkasse)}
    \Def{3 - Internes Schema}{Beschreibt die internen Speicherstrukturen einer Datenbank, ist zudem nicht sichtbar vom Anwender}

    \subsection{Konzeptionelles Schema}
    Das konzeptionelle Schema regelt:
    \Def{Datenunabhängigkeit der Anwendung}{Benutzung der Daten ohne Bezug auf Details der Speicherung, optimierungsfähig.}\\
    \DApp{}{Nur logische (abstrakte Sicht auf die Daten)} \\
    \DApp{}{In anderen Kontexten auch: Datenabstraktion, Datenkapselung oder Information hiding genannt.} \\
    \Def{Anwendungsneutralität der Datenbank}{Offen für die Wiederverwendung durch neue Anwendungen.} \\
    \DApp{}{Gibt ein Strukturen Schema für Anwendungen, vermeidet so Redundanz (optimierend).} \\
    \DApp{}{Anwendungsspezifische Beschreibungen sind nicht im konzeptionellen Schema verankert}

    \subsection{Datenmodellierung}
    \begin{center}
        \begin{tabular}{ c | c }
            Miniwelt & Objekte, Begriffe, Objektbezeichnungen \\
            \hline
            Konzeptioneller Entwurf & E/R-Modell\\
            Konzeptionelles Schema & E/R-Diagramm \\
            \hline
            Logischer Entwurf & Relationales Datenmodell \\
            Logisches (Konzeptionelles) Schema & Relationelles DB-Schema
        \end{tabular}
    \end{center}

    \newpage
    \section{Grundlagen des Entity/Relationship-Modells}
    \subsection{Entities und Attribute}
    \Def{Entity}{Zu beschreibendes Objekt} \\ \\
    Entities sind/haben:
    \begin{itemize}
        \item[1)] Eigenständige Existenz
        \item[2)] Identifizierbar (= unterscheidbar)
        \item[3)] Beschreibbar (Merkmale, Attribute)
        \item[4)] Relevant (für Anwendungswelt)
    \end{itemize}
    \Def{Entity-Typ}{Beschreibung gleichartiger Objekte}
    \Def{Attribut}{Eigenschaft/Merkmal von Objekten eines Entity-Typs}
    \Def{Schlüssel-Attribut}{Attribut oder Kombination aus Attributen die ein Eintity-Typ Identifiziert}
    \Def{Entity-Menge}{Menge gleichartiger Objekte}

    \subsection{Relationships}
    \Def{Relationship}{Beziehung zwischen zwei oder mehr Objekten}
    \Def{Relationship-Typ}{Beschreibung gleichartiger Beziehungen} \\
    \DApp{}{Relationship-Typen können auch Attribute besitzen}
    \Def{Relationship-Menge}{Menge gleichartiger Beziehungen}

    \newpage
    \subsection{Symbole}
        \begin{tabular}{c c}
            \begin{tikzpicture}[auto,node distance=1.5cm]
                \node[entity] (node1) {Entity}
            \end{tikzpicture} &
            \begin{tikzpicture}[auto,node distance=1.5cm]
                \node[weak entity] (node1) {Schwacher Entity Typ}
            \end{tikzpicture} \\ \\
            \begin{tikzpicture}[auto,node distance=1.5cm]
                \node[relationship] (node1) {Rel. Typ}
            \end{tikzpicture} &
            \begin{tikzpicture}[auto,node distance=1.5cm]
                \node[ident relationship] (node1) {Ident. Rel. Typ}
            \end{tikzpicture}
        \end{tabular} \\ \\ \\
    \begin{center}
    \begin{tikzpicture}[auto,node distance=1.5cm]
    \node[entity] (node1) {Entity}
    [grow=right,sibling distance=3.5cm]
    child [grow=down,level distance=2cm, xshift=-3cm] {node[attribute] {\textbf{Schlüssel-Attribut}}}
    child [grow=right,level distance=2cm, yshift=-3cm] {node[multi attribute] {\textbf{Mehrwertiges Attribut}}}
    child {node[derived attribute] {Abgeleitetes Attribut}}
    child [grow=up,level distance=2cm] {node[attribute] {Zusammengesetztes Attribut} [grow=up,sibling distance=2cm] child {node[attribute] {ZA 1.}} child {node[attribute] {ZA 2.}} child {node[attribute] {ZA 3.}}}
    child [grow=left,level distance=4cm] {node[attribute] {Attribut}};
    \end{tikzpicture}
    \end{center}
    
    \newpage
    \subsection{Kardinalitäten der Beziehungen}

    \noindent Totale-Teilnahme von E1 in R: \\

    \begin{tikzpicture}[auto,node distance=1.5cm]
        \node[entity] (node1) {E1};
        \node[relationship] (node2) [right = of node1] {R};
        \node[entity] (node3) [right = of node2] {E2};
        \path (node1) edge[total] (node2);
        \path (node3) edge (node2);
    \end{tikzpicture} \\
    Die Notation der Totalen-Teilnahme ist in den jeglichen Notationen, zur daarstellung von ER sowie EER Diagrammen, anwendbar.
    \KW{Chen-Notation} \\
    Maximal ein E1 ist in Relation R mit maximal einem E2: \\

    \begin{tikzpicture}[auto,node distance=1.5cm]
        \node[entity] (node1) {E1};
        \node[relationship] (node2) [right = of node1] {R};
        \node[entity] (node3) [right = of node2] {E2};
        \path (node1) edge node [draw=none, yshift=0.3cm, xshift=-0.5cm] {\textbf{1}} (node2);
        \path (node3) edge node [draw=none, yshift=0.3cm] {\textbf{1}} (node2);
    \end{tikzpicture}
    \\ \\
    Undefiniert viele E1 sind in Relation R mit undefiniert vielen E2: \\

    \begin{tikzpicture}[auto,node distance=1.5cm]
        \node[entity] (node1) {E1};
        \node[relationship] (node2) [right = of node1] {R};
        \node[entity] (node3) [right = of node2] {E2};
        \path (node1) edge node [draw=none, yshift=0.3cm, xshift=-0.5cm] {\textbf{N}} (node2);
        \path (node3) edge node [draw=none, yshift=0.3cm] {\textbf{M}} (node2);
    \end{tikzpicture}
    \\ \\
    Undefiniert viele E1 sind in Relation R mit genau einem E2: \\

    \begin{tikzpicture}[auto,node distance=1.5cm]
        \node[entity] (node1) {E1};
        \node[relationship] (node2) [right = of node1] {R};
        \node[entity] (node3) [right = of node2] {E2};
        \path (node1) edge node [draw=none, yshift=0.3cm, xshift=-0.5cm] {\textbf{N}} (node2);
        \path (node3) edge node [draw=none, yshift=0.3cm] {\textbf{1}} (node2);
    \end{tikzpicture}
    
    \KW{(Min, Max)-Notation}

    %node [draw=none, yshift=0.3cm, xshift=-0.5cm] {hi}

    \newpage
    \section{Das erweiterte ER-Modell (EER)}
    \subsection{Grundlagen des EER-Modells}
    \Def{Klassen}{Eine Entity-Menge} \\
    \DApp{}{\textbf{Unterklassen} sind Teilmengen einer Klasse} \\
    \DApp{}{\textbf{Oberklassen} sind Teilmengen einer Klasse} \\
    \noindent\large$\forall \; a \in \text{E2},\; b \in \text{E3}: a \notin \text{E3} \land b \notin \text{E2} \land a, b \in \text{E1}$
    \begin{center}
    \begin{tikzpicture}[auto,node distance=1.5cm]
        \node[entity] (node1) {E1};
        \node[circle] (node0) [below=of node1]{\huge d};
        \node[entity] (node2) [below left=of node0] {E2};
        \node[entity] (node3) [below right=of node0] {E3};
        \path (node2) edge node [pos=0.5, draw=none, yshift=0cm, xshift=0, rotate=38] {\huge$\subset$} (node0);
        \path (node3) edge node [pos=0.5, draw=none, yshift=-0.1cm, xshift=0.1cm, rotate=140] {\huge$\subset$} (node0);
        \path (node1) edge (node0);
    \end{tikzpicture}
    \end{center}
    \\ \\
    \noindent\large$\forall \; a \in \text{E1}: a \in \text{E2} \veebar a \in \text{E3}$
    \begin{center}
        \begin{tikzpicture}[auto,node distance=1.5cm]
            \node[entity] (node1) {E1};
            \node[circle] (node0) [below=of node1]{\huge d};
            \node[entity] (node2) [below left=of node0] {E2};
            \node[entity] (node3) [below right=of node0] {E3};
            \path (node2) edge node [pos=0.5, draw=none, yshift=0cm, xshift=0, rotate=38] {\huge$\subset$} (node0);
            \path (node3) edge node [pos=0.5, draw=none, yshift=-0.1cm, xshift=0.1cm, rotate=140] {\huge$\subset$} (node0);
            \path (node1) edge[total] (node0);
        \end{tikzpicture}
    \end{center}
    \\ \\
    \noindent\large$\forall \; a \in \text{E2},\; b \in \text{E3}: (a \in \text{E3} \lor b \in \text{E2}) \land a, b \in \text{E1}$
    \begin{center}
        \begin{tikzpicture}[auto,node distance=1.5cm]
            \node[entity] (node1) {E1};
            \node[circle] (node0) [below=of node1]{\huge o};
            \node[entity] (node2) [below left=of node0] {E2};
            \node[entity] (node3) [below right=of node0] {E3};
            \path (node2) edge node [pos=0.5, draw=none, yshift=0cm, xshift=0, rotate=38] {\huge$\subset$} (node0);
            \path (node3) edge node [pos=0.5, draw=none, yshift=-0.1cm, xshift=0.1cm, rotate=140] {\huge$\subset$} (node0);
            \path (node1) edge (node0);
        \end{tikzpicture}
    \end{center}
    \\ \\
    \noindent\large$\forall \; a \in \text{E1}: a \in \text{E2} \lor a \in \text{E3}$
    \begin{center}
        \begin{tikzpicture}[auto,node distance=1.5cm]
            \node[entity] (node1) {E1};
            \node[circle] (node0) [below=of node1]{\huge o};
            \node[entity] (node2) [below left=of node0] {E2};
            \node[entity] (node3) [below right=of node0] {E3};
            \path (node2) edge node [pos=0.5, draw=none, yshift=0cm, xshift=0, rotate=38] {\huge$\subset$} (node0);
            \path (node3) edge node [pos=0.5, draw=none, yshift=-0.1cm, xshift=0.1cm, rotate=140] {\huge$\subset$} (node0);
            \path (node1) edge[total] (node0);
        \end{tikzpicture}
    \end{center}
    
    
    \KW{Vererbung}
    \begin{itemize}
        \item Entities in einer Unterklasse "erben" alle Attribute aus der Oberklasse.
        \item Entities der Unterklasse "erben" alle Beziehungen aus der Oberklasse.
        \item Ein Schlüssel der Oberklasse ist immer auch Schlüssel der Unterklassen. 
        \item Unterklassen können ggf. weitere alternative Schlüssel (nur für die Unterklasse!) ergänzen.
    \end{itemize}

    \noindent \TODO{Union EER (Kategorie)} \\
    \TODO{Krähenfuß} \\

    \newpage
    \section{Das Relationale Datenmodell}
    \subsection{Grundlagen des Relationenmodells}

    \noindent Das Prinzip des Relationenmodells liegt darin die Abfrage von Daten über inhaltliche kretieren zu veranlassen, im gegensatz zu über Speicherstrukturen. \\\\
    Das Relationenmodell benötigt eine einheiltliche logische Datenstruktur. \\ \\
    Alle Daten sind in Tabellen strukturiert, eine solche Tabelle wird als Relation bezeichnet.\\ D.h. Relationale Datenbank = Menge von Relationen.
    
    \Def{Kardinalität}{Anzahl der Spalten in einer Tabelle im relationalen Datenmodell.}
    \newpage
    \section{Mapping}
    \subsection{Grundlagen beim Mapping}

    \noindent Mapping ist das Abbilden eines ER/EER modell's auf das relationale Datenmodell.
    
    \newpage
    \section{Normalisierung}
    \subsection{Grundlagen der Normalisierung}
    \Def{Superschlüssel}{Alle Attribute einer Relation sind funktional Abhängig von einem Attribut, dem Superschlüssel}
    \subsection{Erste Normalform}
    \noindent\DApp{Def.}{Eine Relation ist in erster Normalform (1NF), wenn sie nur atomare Attributwerte besitzt.}
    \\ \\
    Für eine Relation (Tabelle) in 1NF gilt:
    \begin{itemize}
        \item Jeder Attributwert ist atomar.
        \item Es gibt keine zwei gleichen Tupel (Zeilen).
        \item Die Reihenfolge der Tupel ist unerheblich.
        \item Jeder Attributwert (in einer Spalte) gehört zum Wertebereich des Attributs.
        \item Jedes Attribut (jede Spalte) hat einen eindeutigen Namen.
        \item Die Reihenfolge der Attribute ist unerheblich.
    \end{itemize}

    \Def{Atomar}{Es gibt maximal einen wert in einem Feld.}

    \subsection{Zweite Normalform}
    \noindent\DApp{Def.}{Einen Relation ist in zweiter Normalform (2NF), wenn sie in erster Normalform ist und alle Nicht-Schlüsselattribute voll funktional von jedem Schlüsselkandidaten abhängen.}
    \\ \\
    Ziel der 2NF ist die eliminierung "partieller Abhängigkeiten" der Form: \\ \\
    R(\underline{A,B},C,D): A,B $\rightarrow$ C u. B $\rightarrow$ D $\xRightarrow{\text{In 2NF:} \; \;}$ R1(\underline{A,B[R2]},C), R2(\underline{B},D)
    
    \subsection{Dritte Normalform}
    \noindent\DApp{Def.}{Eine Tabelle ist in dritter Normalform (3NF), wenn kein Nicht-Schlüsselattribut transitiv abhängig von einem Schlüsselkandidaten ist.}
    \\ \\
    Ziel der 3NF ist die eliminierung "transitiver Abhängigkeiten" der Form: \\ \\
    R(\underline{A},B,C,D): A $\rightarrow$ B, C u. B $\rightarrow$ C $\xRightarrow{\text{In 3NF:} \; \;}$ R1(\underline{A,B[R2]},D), R2(\underline{B},C)

    \subsection{Boyce-Codd Normalform}
    \noindent\DApp{Def.}{Eine Relation ist in Boyce-Codd-Normalform (BCNF), wenn jede Determinante einer funktionalen Abhängigkeiten ein Superschlüssel ist.}
    \\ \\
    Ziel der BCNF ist es alle Schlüsselkandidaten zu Superschlüssel zu machen : \\ \\
    R(\underline{A,B},C): A,B $\rightarrow$ C u. C $\rightarrow$ B $\xRightarrow{\text{In 3NF:} \; \;}$ R1(\underline{A,C[R2]}), R2(\underline{C},B)
 \end{document}